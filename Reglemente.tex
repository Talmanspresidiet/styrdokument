\documentclass{styrdokument}

\title{Reglemente}
\date{\today}
%\link{\href{https://github.com/Talmanspresidiet/styrdokument}{\texttt{https://github.com/Talmanspresidiet/styrdokument}}}

\begin{document}

\section{Medlemskap}
\subsection{Hedersmedlemmar}

\? Sektionens hedersmedlemmar är
\begin{itemize}
    \item Valen Åke som strandade i Träslövsläge.
    \item Schrödingers katt, kanske.\\
    {\footnotesize\itshape Under sektionsmötet 2013–05–07 genomfördes en omröstning om nämnda katts medlemskap. Omröstningen skedde enligt mötets önskan genom sluten votering, och rösterna placerades i ett kuvert, som sedan förseglades.
    Schrödingers katt är därmed kanske hedersmedlem.}
\end{itemize}

\section{Sektionsmötet}
\subsection{Åligganden}

\? Det åligger dessutom sektionsmötet att innan utgången av läsperiod 1
\begin{itemize}
    \item anta sammanträdesordning på förslag av föregående års talman,
    \item fastställa verksamhetsplan för studienämnden.
\end{itemize}

\? Det åligger dessutom sektionsmötet att innan utgången av läsperiod 4
\begin{itemize}
    \item fastställa preliminär verksamhetsplan för sektionsstyrelsen,
    \item fastställa preliminär budget för sektionen.
\end{itemize}

\? Det åligger dessutom sektionsmötet att innan utgången av läsperiod 1 välja
\begin{itemize}
    \item Balnågonting
    \item Frisörer
    \item Kräldjursvårdare
    \item Sektionsnörd
    \item årskursrepresentant i studienämnden
\end{itemize}

\? Det åligger dessutom sektionsmötet att innan utgången av läsperiod 2 välja
\begin{itemize}
    \item FARM
    \item FIF
    \item FnollK
\end{itemize}

\? Det åligger dessutom sektionsmötet att innan utgången av läsperiod 3 välja
\begin{itemize}
    \item Bakisclubben
    \item Bilnissar
    \item Blodgrupp
    \item Fabiola
    \item Fanfareri
    \item Finform
    \item Foton
    \item Game Boy
    \item Mastermottagningsansvarig
    \item Piff och Puff
    \item Spidera
    \item Sångförmän
\end{itemize}

\? Det åligger dessutom sektionsmötet att innan utgången av läsperiod 4 välja
\begin{itemize}
    \item Djungelpatrullen
    \item Dragos
    \item F6
    \item Focumateriet
    \item Studienämnden, exkl. årskursrepresentant
    \item fristående ledamöter i JämF
    \item resterande ledamöter i Sektionsstyrelsen
\end{itemize}

\subsection{Utlysning}
\? Kallelse samt slutgiltig föredragningslista ska utöver anslag tillsändas sektionsmedlemmar, revisorer, inspektor och kårledningen.

\? Datum för ordinarie sektionsmöten fastställs av talmanspresidiet på ett styrelsemöte i samråd med styrelsen.

\subsection{Personval}
\? Vid personval där det finns fler kandidater än poster ska sluten votering användas.

\? Vid personval där två eller fler kandidater får lika röstetal som hade lett till val ska ny votering göras med dessa kandidater.
Skulle på nytt lika röstetal uppkomma skiljer lotten.

\? Vid personval där godkänd gruppnominering enligt \cref{valb:grupp} avgetts ska nominerade anses valda enligt denna om sektionsmötet enligt valberedningens nominering väljer motsvarande förtroendeposter.
I annat fall sker personval till dessa poster enligt vanlig ordning.

\section{Sektionsstyrelsen}
\subsection{Sammansättning}
\? Sektionsstyrelsen består av följande poster:
\begin{itemize}
    \item Sektionsordförande
	\item Vice sektionsordförande
	\item Sektionskassör
	\item Sekreterare
	\item Skyddsombud
	\item Informationsansvarig
	\item Ordförande i studienämnden
	\item Ordförande i FnollK
	\item Ordförande i F6
	\item Ordförande i FARM
	\item Ordförande i Focumateriet
	\item Ordförande i Djungelpatrullen
\end{itemize}
varav samtliga är förtroendeposter.
			
\? Vice ordförande i respektive kommitté samt studienämnden är suppleant i sektionsstyrelsen.
		
\? Suppleant övertar sin ordinaries befogenheter vid styrelsemöten vid bortfall, undantaget då styrelsemötet hålls bakom stängda dörrar.

\subsection{Styrelsemöten}
\? Kallelse till styrelsemöte ska senast två dagar innan mötet skickas till ordinarie ledamöter av sektionsstyrelsen, medlem av kommitté och nämnd, revisorer, talmanspresidiet, valberedningen samt övriga berörda sektionsföreningsmedlemmar och funktionärer.
		
\? Varje medlem har rätt att få fråga behandlad på styrelsemöte.
Sådan fråga skickas till styrelsen senast tre dagar innan mötet.

\subsubsection{Rättigheter}
\? Ordinarie ledamöter av sektionsstyrelsen har närvaro-, yttrande-, förslags- och rösträtt. 

\? Talman, revisor och ledamöter av valberedningen har närvaro-, yttrande- och förslagsrätt.
		
\? Ledamot av kommitté och studienämnden har närvaro- och yttranderätt.
		
\? Mötesdeltagare kan ej adjungeras in med rösträtt.

\subsubsection{Bakom stängda dörrar}
\? Sektionsstyrelsen kan, om synnerliga skäl föreligger, för visst ärende med minst \sfrac{2}{3} majoritet besluta att överläggning sker bakom stängda dörrar.

\? Enbart ordinarie ledamöter av sektionsstyrelsen äger närvarorätt vid möte bakom stängda dörrar.
Övriga deltagare kan adjungeras in.

\? Endast beslutsprotokoll fört då mötet hålls bakom stängda dörrar ska anslås.

\? Det som diskuteras bakom stängda dörrar får ej föras vidare till tredje part. 

\subsection{Åligganden}
\? Det åligger sektionsstyrelsen
\begin{aligganden}
    \item verka för sammanhållningen mellan sektionsmedlemmarna och deras gemensamma intressen.
	\item leda sektionens arbete.
	\item verkställa och övervaka genomförandet av sektionsmötesbeslut.
	\item framlägga budget till sektionsmötet.
	\item planera sektionens framtida inriktning och verksamhet.
	\item fatta beslut i de ärenden som framlägges till sektionsstyrelsen.
	\item varje år tillsammans med studienämnden utse sektionens representanter i styrelser och kommittéer inom högskolan.
	    Ledamöterna i programråden fastställs av sektionsmötet.
	\item i samråd med de som önskar söka sektionsstyrelsen ta fram en preliminär verksamhetsplan och presentera denna på sektionsmöte innan utgången av läsperiod 4.
\end{aligganden}

\? Det åligger sektionsstyrelsens ordförande:
\begin{aligganden}
    \item tillse att sektionens beslut verkställs.
    \item föra sektionens talan då något annat ej stadgats eller beslutats.
    \item leda och övervaka arbetet inom sektionsstyrelsen.
    \item vara sektionens representant i kårledningsutskottet.
    \item tillse att det finns representanter från sektionsstyrelsen i programråden för sektionens program.
\end{aligganden}

\? Det åligger sektionsstyrelsens vice ordförande:
\begin{aligganden}
    \item i samråd med styrelsen och övriga sektionsaktiva upprätta sektionens verksamhetsberättelse.
\end{aligganden}

\? Det åligger sektionsstyrelsens  kassör:
\begin{aligganden}
    \item fortlöpande kontrollera kommittéernas samt eventuella sektionföreningars räkenskaper och bokföring.
    \item genom Chalmers studentkår uppbära sektionsavgiften.
    \item i samråd med sektionsstyrelsen upprätta preliminärt budgetförslag till första ordinarie höstmötet.
    \item till varje sektionsmöte kunna redogöra för sektionens ekonomiska ställning.
\end{aligganden}

\? Det åligger sektionsstyrelsens sekreterare:
\begin{aligganden}
    \item föra protokoll vid styrelsemöten och senast två läsdagar efter möte överräcka renskrivet protokoll till ordföranden.
    \item tillse att protokoll från styrelsemöten anslås.
    \item tillse att sektionens stadgar, reglemente och förordningar är aktuella och efterlevs.
\end{aligganden}

\? Det åligger sektionsstyrelsens skyddsombud:
\begin{aligganden}
    \item vara sektionens studerandearbetsmiljöombud samt vara sektionens jämlikhetsansvarige.
    \item lyda under tystnadsplikt i sitt uppdrag.
    \item tillvarata sektionsmedlemmarnas intressen i skyddsfrågor, jämlikhets- och jämställdhetsfrågor och samarbeta med Chalmers studentkårs kontaktperson samt högskolans skyddsombud.
    \item deltaga på studienämndens möten då arbetsmiljöfrågor behandlas.
    \item vara sektionsstyrelsens representant i JämF.
\end{aligganden}

\? Det åligger sektionsstyrelsens informationsansvariga:  
\begin{aligganden}
    \item sköta kontakt med sektionen samt uppdatera sektionens hemsida. 
    \item tillse att material som inkommer till sektionen anslås eller på annat sätt förmedlas till berörda parter.
    \item vara sektionsstyrelsens representant i Spidera.
\end{aligganden}    

\section{Studienämnden}
\subsection{Sammansättning}
\? Studienämnden består av följande poster:
\begin{itemize}
    \item Ordförande
	\item Vice ordförande
	\item Kassör
	\item Sekreterare
	\item Kandidatansvarig
	\item Masteransvarig
	\item Årskursrepresentant åk. 1
	\item Veckobladerist
	\item Matansvarig.
	\end{itemize}
varav ordförande, vice ordförande och kassör är förtroendeposter.

\subsection{Åligganden}
\? Det åligger studienämnden:
\begin{aligganden}
    \item ansvara för utvecklandet av utbildningsbevakningen på sektionen.
    \item inför sektionen svara för att teknologerna på sektionens programs intressen i studiefrågor och studiemiljö bevakas på ett tillfredsställande sätt.
    \item tillse att det finns representant från studienämnden i programråden för sektionens program.
    \item på verksamhetsårets första sektionsmöte presentera en verksamhetsplan för det kommande läsåret.
\end{aligganden}

\? Det åligger studienämndens ordförande:
\begin{aligganden}
    \item leda studienämndens verksamhet.
    \item kalla studienämnden till sammanträde.
    \item representera studienämnden i sektionsstyrelsen.
    \item i studie- och studiemiljöfrågor representera sektionen och föra dess talan.
    \item representera sektionen i Utbildningsutskottet, UU.
\end{aligganden}

\? Det åligger studienämndens vice ordförande:
\begin{aligganden}
    \item ansvara för studiesociala evenemang.
\end{aligganden}

\? Det åligger studienämndens kassör:
\begin{aligganden}
    \item mot revisorerna och sektionskassören kontinuerligt redovisa den ekonomiska situationen. 
\end{aligganden}

\? Det åligger studienämndens sekreterare:
\begin{aligganden}
     \item tillse att protokoll förs på studienämndens möten.
\end{aligganden}

\section{Kommittéer}
\subsection{Förteckning}
\? Sektionens kommittéer är:
\begin{itemize}
    \item FARM
	\item FnollK
	\item F6
	\item Djungelpatrullen
	\item Focumateriet
\end{itemize}

\subsection{Allmänna åligganden}
\? Det åligger varje kommitté:
\begin{aligganden}
    \item inom en läsperiod efter dess inval presentera en verksamhetsplan för det kommande verksamhetsåret för sektionsstyrelsen. 
\end{aligganden}

\? Det åligger ordföranden i varje kommitté:
\begin{aligganden}
    \item leda kommitténs arbete.
    \item fungera som en kontaktlänk mellan kommittén och andra organ samt företräda kommittén i sektionsstyrelsen.
\end{aligganden}

\? Det åligger kassören i varje kommitté:
\begin{aligganden}
    \item mot revisorerna och sektionskassören kontinuerligt redovisa för den ekonomiska situationen.
\end{aligganden}

\subsection{FARM}
\? FARM är sektionens arbetsmarknadsgrupp.

\? FARM består av följande poster:
\begin{itemize}
    \item Ordförande
 	\item Vice ordförande
	\item Kassör
	\item 5 ledamöter
\end{itemize}
varav ordförande, vice ordförande och kassör är förtroendeposter.

\? Det åligger FARM:
\begin{aligganden}
    \item arrangera studiebesök och branschkvällar.   
    \item informera företag om sektionens program och deras fördelar.
\end{aligganden}

\subsection{FnollK}
\? FnollK är sektionens mottagningskommitté.

\? FnollK består av följande poster:
\begin{itemize}
    \item Ordförande
	\item Vice ordförande
	\item Kassör
	\item 4 ledamöter
\end{itemize}
varav ordförande, vice ordförande och kassör är förtroendeposter.
		
\? Det åligger FnollK:
\begin{aligganden}
    \item genomföra en värdig mottagning i enlighet med kårens och sektionens anda i samråd med relevanta organ.
    \item arrangera aktiviteter för Nollan som syftar till att införliva dem i livet som teknolog vid sektionens program både vad gäller studier och det studiesociala livet.
    \item tillse att valen Åke bär nollbricka om gamble så tycker.
    \item tillse att Schrödingers katt bär nollbricka.
    \item tillse att Schrödingers katt ej bär nollbricka.
\end{aligganden}

\? Det åligger FnollK:s ordförande:
\begin{aligganden}
    \item representera sektionen i Chalmers studentkårs samarbetsorgan för mottagningen, MoS.
\end{aligganden}

\subsection{F6}
\? F6 är sektionens sexmästeri.

\? F6 består av följande poster:
\begin{itemize}
	\item Ordförande, Sexmästare
	\item Vice ordförande, Sexreterare
	\item Kassör
	\item 6 ledamöter
\end{itemize}
varav ordförande, vice ordförande och kassör är förtroendeposter.

\? Det åligger F6:
\begin{aligganden}
    \item minst en gång per läsperiod anordna gasque.
    \item ansvara för kalas- och tentamensfestlighetsverksamhet på sektionen.
    \item vara ett komplement till FnollK under mottagningen.
\end{aligganden}

\? Det åligger F6:s ordförande, Sexmästaren:
\begin{aligganden}
    \item vara sektionens representant i Gasquerådet om F6 så beslutar.
\end{aligganden}

\subsection{Djungelpatrullen}
\? Djungelpatrullen är sektionens rustmästeri och PR-förening.

\? Djungelpatrullen består av följande poster:
\begin{itemize}
    \item Ordförande, Överste
	\item Vice ordförande, Rustmästare
	\item Kassör, Skattmästare
	\item 7 adjutanter
\end{itemize}
varav ordförande, vice ordförande och kassör är förtroendeposter.

\? Det åligger Djungelpatrullen:
\begin{aligganden}
    \item tillse att sektionshelgonet vördas på ett hedersamt sätt av alla sektionens medlemmar.   
    \item ansvara för att det ordnas arrangemang för medlemmarna på sektionen.
        Dessa ska hållas i en anda som ökar sammanhållningen på sektionen och ökar kontakten över årskursgränserna.
    \item vara ett komplement till FnollK under mottagningen. 
    \item sköta det löpande underhållet av sektionens lokaler och egendom.
    \item vårda sektionens traditioner.
    \item tillse att sektionens förstaårsstudenter städar sektionslokalen.
\end{aligganden}

\? Det åligger Djungelpatrullens vice ordförande, Rustmästaren:
\begin{aligganden}
    \item leda renoverings- och underhållsarbeten i sektionens lokaler.
    \item ansvara för uthyrning av Focus inventarier.
\end{aligganden}

\subsection{Focumateriet}
\? Focumateriet är sektionens focumateri.

\? Focumateriet består av följande poster:
\begin{itemize}
    \item Ordförande, Kapten
	\item Vice ordförande, Automatpirat
	\item Kassör, Kistväktare
	\item 5 ledamöter
\end{itemize}
varav ordförande, vice ordförande och kassör är förtroendeposter.

\? Det åligger Focumateriet:
\begin{aligganden}
    \item handha Focumaten.
    \item efter sektionsstyrelsens bestämmande handha sektionens automater och annan elektronisk utrustning.
\end{aligganden}

\section{Sektionsföreningar}
\subsection{Förteckning}
\? Sektionens sektionsföreningar är:
\begin{itemize}
    \item Fabiola
	\item FIF
	\item Foton
	\item Game Boy
\end{itemize}

\subsection{Allmänna åligganden}
\? Det åligger ordförande i sektionsförening:
\begin{aligganden}
    \item leda sektionsföreningens arbete.
    \item fungera som en kontaktlänk mellan sektionsföreningen och andra organ.
\end{aligganden}

\? Det åligger ekonomiskt ansvarig i sektionsförening:
\begin{aligganden}
    \item mot revisorerna och sektionskassören kontinuerligt redovisa för den ekonomiska situationen.
\end{aligganden}

\subsection{Fabiola}
\? Fabiola är sektionens förening för kvinnor och ickebinära.

\? Fabiola består av 10 ledamöter varav en väljs internt till ordförande. Fabiola har inga förtroendeposter.
		
\? Det åligger Fabiola:
\begin{aligganden}
    \item arrangera evenemang riktade mot kvinnor och ickebinära på sektionen.
\end{aligganden}   %ändrad enl sektmöte lp4 20/21

\subsection{FIF}
\? FIF är sektionens idrottsförening.

\? FIF består av följande poster:
\begin{itemize}
    \item Ordförande
	\item Vice ordförande
	\item Kassör
	\item 6 ledamöter
\end{itemize}
varav ordförande, vice ordförande och kassör är förtroendeposter.
Ordförande och kassör är ekonomiskt ansvariga.

\? Det åligger FIF:
\begin{aligganden}
    \item främja idrottskulturen på sektionen genom att arrangera regelbundna träningar.
    \item handha idrottsmaterial samt sköta utlåning av denna till sektionsmedlemmar.
\end{aligganden}

\? Det åligger FIF:s ordförande:
\begin{aligganden}
      \item tillse att kontakt med andra sektioners idrottsföreningar uppehålls.
\end{aligganden}

\subsection{Foton}
\? Foton är sektionens fotoförening.

\? Foton består av 6 ledamöter varav en väljs internt till ordförande. Foton har inga förtroendeposter.
		
\? Det åligger Foton:
\begin{aligganden}
    \item löpande dokumentera sektionens arrangemang.
    \item i samband med arbetsmarknadsmässor arrangera porträttfotografering.
    \item stödja sektionens kommittéer och nämnder vid behov av filmning eller fotografering.
\end{aligganden}
	
\subsection{Game Boy}
\? Game Boy är sektionens brädspelsförening.

\? Game Boy består av 6 Game Boys. Game Boy har inga förtroendeposter.
	    
\? Det åligger Game Boy:
\begin{aligganden}
    \item ta hand om de sällskapsspel som finns på Focus.
	\item främja brädspelsverksamheten på sektionen.
\end{aligganden}

\section{Funktionärer}
\subsection{Förteckning}
\? Sektionens funktionärer är:
\begin{itemize}
    \item Finform
	\item Sångförmännen
	\item Revisorer
	\item Dragos
	\item Fanfareriet
	\item Bilnissar
	\item Blodgruppen
	\item Kräldjursvårdare
	\item Dumvästinnehavare
	\item Bakisclubben
	\item Spidera
	\item Sektionsnörd
	\item Balnågonting
	\item Piff och Puff
	\item JämF
	\item Mastermottagningsansvarig
	\item Frisörer
\end{itemize}

\subsection{Finform}
\? Finform är sektionens informationsskrift och ska på ett lättillgängligt sätt presentera intressanta fakta, skämt och skvaller. 

\? Finform består av följande poster:
\begin{itemize}
    \item Chefredaktör tillika ansvarig utgivare
	\item Kassör
	\item 8 redaktörer
\end{itemize}

\? Ansvarig utgivare för Finform tillträder efter inregistrering enligt gällande lag.

\? Det åligger Finformredaktionen:
\begin{aligganden}
    \item producera minst 4 nummer av Finform per läsår, varav minst 2 på hösten och minst 2 på våren.
    \item ansvara för tryckning och distribution av Finform.
\end{aligganden}

\? Det åligger Finforms chefredaktör:
\begin{aligganden}
    \item leda Finforms arbete.
    \item tillse att Finform inte agerar olagligt, kränkande eller på annat olämpligt sätt.
    \item tillse att Finform agerar på ett lämpligt sätt för att vara sektionens officiella informationsskrift.
\end{aligganden}

\? Det åligger Finforms kassör:
\begin{aligganden}
    \item tillsammans med chefredaktören ansvara för att Finforms ekonomiska anslag används inom av sektionsstyrelsen fastställd ram.
\end{aligganden}

\subsection{Sångförmän}
\? Sångförmännens syfte är att förvalta och bevara sektionens sångtraditioner.

\? Sångförmännen är 6 till antalet.

\subsection{Dragos}
\? Dragos är sektionens högste beskyddare, och utövar Fanfareriets högsta befäl.

\subsection{Fanfareriet}
\? Fanfareriets syfte är att ta hand om sektionens fanor och flaggor.

\? Fanfareriet består av en flaggmarskalk och 2 fanbärare.

\subsection{Bilnissar}
\? Bilnissarna består av en ekonomisk bilnisse samt en mekanisk bilnisse.

\? Det åligger Bilnissarna:
\begin{aligganden}
    \item ansvara för de motorfordon som sektionsstyrelsen beslutat om, dock endast sådana som sektionen helt eller delvis förfogar över.
    \item ansvara för uthyrning av dessa fordon, enligt taxa fastställd av sektionsstyrelsen.
\end{aligganden}

\? Det åligger ekonomisk bilnisse:
\begin{aligganden}
    \item vara sektionskassören behjälplig vid ekonomiska ärenden rörande bilnisse.
\end{aligganden}

\? Det åligger mekanisk bilnisse:
\begin{aligganden}
    \item tillse att ovannämnda fordon underhålls och repareras på ett tillfredsställande sätt.
\end{aligganden}

\subsection{Blodgruppen}
\? Blodgruppens syfte är att uppmuntra sektionsmedlemmarna till att lämna blod.

\? Blodgruppen består av fem ledamöter varav en väljs internt till ordförande.

\subsection{Kräldjursvårdare}
\? Kräldjursvårdarens syfte är att ta hand om sektionens slang, Tilde.

\? Det ska finnas en kräldjursvårdare på sektionen.

\? Sektionen ska inte ha kräldjur på Focus.

\subsection{Dumvästinnehavare}
\? Det ska väljas en Dumvästinnehavare på varje sektionsmöte.

\? Den, som med avsikt att erhålla Dumvästen utfört dumheter bör ej vara kvalificerad till denna.
		
\? Kriminella handlingar är ej kvalificerade till Dumvästen, såvida inte sektionsmötet anser detta.
		
\? Om ej tillräckligt kvalificerad dumhet nomineras, kvarstår Dumvästen hos innehavaren för tillfället.
		
\? Förteckning över Dumvästinnehavare genom tiderna tillhandahålls av sektionsstyrelsen.

\subsection{Bakisclubben}
\? Bakisclubbens syfte är att främja bakverksamheten på sektionen.

\? Bakisclubben består av 6 bakisar.

\? Det åligger Bakisclubben:
\begin{aligganden}
    \item inför Kanelbullens dag tillse att kanelbullar bakas och görs tillgängliga på Focus för alla sektionsmedlemmar att avnjuta.
\end{aligganden}

\subsection{Spidera}
\? Spideras syfte är att ansvara för sektionens IT.

\? Spidera består av 10 nätmakare samt informationsansvarig från sektionsstyrelsen.
Av dessa väljs en internt till nätmästare i samråd med sektionsstyrelsen.

\? Det åligger Spidera:
\begin{aligganden}
    \item administrera och utveckla Fysikteknologsektionens internetportal.
\end{aligganden}

\? Det åligger Spideras nätmästare:
\begin{aligganden}
	\item leda Spideras arbete.
    \item tillse att ansvarig för av Spidera disponerad hårdvara är medlem i Spidera.
\end{aligganden}

\subsection{Sektionsnörd}
\? Sektionsnördens syfte är att ta hand om allting som rör sektionens prenumeration av Fantomen.

\subsection{Balnågonting}
\? Balnågontings syfte är att anordna bal, middag samt kringaktiviteter som tillhör balen.

\? Balnågonting har 5 medlemmar.

\subsection{Piff och Puff}
\? Piff och Puff är sektionens aktivitetsgrupp, och syftar till att hjälpa sektionen engagera och underhålla fler sektionsmedlemmar, samt hjälpa sektionsmedlemmar med vägledning och tips om hur aktiviteter för sektionsmedlemmar kan genomföras.

\? Piff och Puff består av 4 piffar.
		
\subsection{JämF}
\? JämF är sektionens jämlikhetsråd, och syftar till att arbeta för en mer jämlik sektion där alla känner sig välkomna.
	    
\? JämF består av sektionsstyrelsens skyddsombud, en representant vardera från de kommittéer och sektionsföreningar där intresse för medlemskap finns, samt 6 fristående ledamöter.
Av dessa väljs en internt till ordförande.
	    
\subsection{Mastermottagningsansvarig}
\? Mastermottagningsansvarigs syfte är att vara med och arrangera en mastermottagning vid utbildningsområdet som sektionens kandidatprogram tillhör.
        
\? Det åligger mastermottagningsansvarig
\begin{aligganden}
    \item tillsammans med andra mastermottagningsansvariga vid utbildningsområdet för sektionens kandidatprogram arrangera en mottagning för nya studenter vid relaterade masterprogram.
\end{aligganden}
        
\subsection{Frisörer}
\? Frisörernas syfte är att sköta om sektionens sektionssten, Einsten.
        
\? Frisörerna är till antalet 2.

\section{Intresseföreningar}
\subsection{Förteckning}
\? Sektionens intresseföreningar är
\begin{itemize}
    \item F-spexet \\
        {\itshape F-spexets syftet är att årligen verka för att sätta upp ett spex, och på så sätt sprida spexkulturen i Göteborgsområdet allt medan spexets medlemmar har roligt.}
    \item 3Dteamet \\
        {\itshape 3Dteamets syfte är att verka för att till sina medlemmar tillgängliggöra 3D-skrivare och annan utrustning som föreningen har att tillgå samt att administrera och underhålla utrustning och hemsida.}
    \item FyS \\
        {\itshape Fysikteknologsektionens spelförenings syfte är att främja intresset av digitala spel på sektionen.}
\end{itemize}

\section{Valberedningen}
% Not: strök valberedningens ordf och vice val internt, det löser sig självt

\subsection{Åligganden}
\? Valberedningen ansvarar utöver åliggande i stadga dessutom för nomineringar till
\begin{itemize}
    \item Studienämnden
    \item Djungelpatrullen
    \item F6
    \item Focumateriet
    \item FARM
    \item FnollK
\end{itemize}

\subsection{Nomineringsbeslut}
\? Ledamot i valberedningen får inte delta i beslut som rör ledamoten själv eller annan närstående, eller om annat jäv föreligger.

\? Poster utöver förtroendeposter i organ kan nomineras i grupp av valberedningen.
Sådan gruppnominering ska fastställas av sektionsstyrelsen senast tre läsdagar innan sektionsmötet då valet äger rum.
\label{valb:grupp}

\? Sektionsstyrelsen bedömer då valberedningens arbete. 
Avslag av gruppnomineringen får ske om synnerliga skäl föreligger.
Beslut om avslag anslås omedelbart med motivering.

% TODO ta bort den här skiten?

\section{Talmanspresidiet}
\subsection{Åligganden}
\? Det åligger talman:
\begin{aligganden}
    \item tillse att sektionsmöten utlyses och fortlöper stadgeenligt.
    \item opartiskt leda sektionsmötet och särskilt bevaka demokratiska principer.
\end{aligganden}

\? Det åligger talmanspresidiets sekreterare:
\begin{aligganden}
    \item föra protokoll under sektionsmötet och tillse att det justeras och anslås i tid.
\end{aligganden}

% vaskar det om vakans val på mötet till mötesordningen

\section{Ekonomi och revision}
\subsection{Verksamhetsår}
\? Om inget annat anges har ett organ samma verksamhetsår som sektionen.

\? Följande organs verksamhetsår löper från 1 januari till 31 december:
\begin{itemize}
    \item FARM
    \item FnollK
    \item FIF
\end{itemize} 

\subsection{Redovisning och ansvarsfrihet}
\? Följande organ, utöver studienämnden och kommittéer, upprättar egen löpande redovisning:
\begin{itemize}
    \item FIF
\end{itemize}

\? Delbokslut och årsredovisning enligt detta avsnitt presenteras för sektionsstyrelsen inom fyra läsveckor efter organets verksamhetsårs slut.

\? Studienämnden presenterar dessutom verksamhetsberättelse vid första ordinarie sektionsmöte efter verksamhetsårets slut.
Handlingarna tillställs enligt vad som stadgats för sektionsstyrelsens årsredovisning.

% TODO deluxe

\subsection{Fonder}
\? Sektionens fonder är:
\begin{itemize}
    \item F-fonden
    \item Focusfonden
\end{itemize}

\? Fonderna handhas av sektionsstyrelsen.

\? Fondernas medel placeras i räntefonder med låg risk.
Ränteavkastningen återinvesteras.

\subsubsection{F-fonden}
\? F-fonden syftar till att säkerställa sektionens likviditet.

\? F-fondens medel kan användas till det som sektionsmötet finner lämpligt, t.ex. renovering av sektionens lokaler, inköp och underhåll av inventarier så som sektionsbil, oförutsedda utgifter och täckande av underskott.

\? Uttag ur F-fonden beslutas av sektionsmöte.
Ärendet ska anges i kallelse.

\? Sektionsstyrelsen får om nödvändigt kortvarigt låna medel ur F-fonden.
I sådant fall behöver fonden ej kompenseras för ränteförlusten.

\? Överskott i sektionens verksamhet tillfaller F-fonden.

\? Om F-fondens värde understiger ett prisbasbelopp ska avsättning göras med minst 15 procent av inbetalda sektionsavgifter tills dess denna nivå uppnås.

\subsection*{Focusfonden}
\? Focusfonden syftar till att möjliggöra större upprustningar av Focus.

\? Focusfondens medel kan användas till renovering av Focus eller inköp av inventarier till Focus.

\? Uttag ur Focusfonden beslutas av sektionsstyrelsen i samråd med Djungelpatrullen.

\? Överskott inom budgetposten ''Focus upprustning'' samt samtliga hyresintäkter för Focus tillfaller Focusfonden.



\section{Styrdokument}
\subsection{Förteckning}
\? Följande övriga styrdokument ska finnas på sektionen:
\begin{itemize}
    \item Arbetsordningar
    \item Sammanträdesordning för sektionsmötet
    \item Förteckning över sektionsmötesbeslut
    \item Ekonomisk policy
    \item Integritetspolicy
    \item Miljö- och hållbarhetspolicy
    \item Lokalpolicy Focus
\end{itemize}

\? Därtill får sektionsstyrelsen förklara ytterligare dokument vara styrdokument.
Komplett förteckning över övriga styrdokument ska tillhandahållas av sektionsstyrelsen.

\subsection{Ändring}
\? Ändring av övrigt styrdokument görs av sektionsstyrelsen med enkel majoritet om inget annat anges.

\? Sektionsstyrelsen ska anslå väsentliga ändringar gjorda i styrdokument.

\subsection{Tillkännagivande}
\? Vad som i stadga, reglemente eller annat styrdokument ska anslås görs så via sektionens officiella kommunikationskanaler, vilka utgörs av sektionens hemsida \href{ftek.se} samt sektionens anslagstavlor.

\subsection{Sekretess}
\? Följande dokument har medlem ej rätt att ta del av:
\begin{itemize}
    \item Incidenthanteringsprotokoll
    \item Handling innehållande personuppgifter
\end{itemize}

\? Därutöver har sektionsstyrelsens skyddsombud tystnadsplikt i sitt uppdrag.

\? Sektionsaktiv får dock hantera sekretessbelagd uppgift i organets ordinarie verksamhet, men inte föra den vidare.

\? Sektionsstyrelsen kan om synnerliga skäl föreligger med enhälligt beslut lämna ut sekretessbelagd uppgift.

\subsection{Tolkningstvister}
\? Tolkning av övrigt styrdokument görs av sektionsstyrelsen med enkel majoritet om inget annat anges.

\? Vid tvist om reglementets tolkning avgörs frågan av inspektor.

% TODO gör overaller till policy

\end{document}

